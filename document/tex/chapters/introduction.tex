% !TEX root = ../../document.tex

\documentclass{subfiles}

\begin{document}

  \chapter{Introducción}
  \label{chap:intro}

    \paragraph{}
    El \emph{Trabajo de Fin de Grado} representa la última fase para la obtención de una titulación de \emph{Grado} en el modelo educativo Español. Para poder que el alumno pueda presentar su trabajo final ante un tribunal, es necesario que este haya completado el resto de créditos de la titulación. Por tanto, el \emph{Trabajo de Fin de Grado} representa la última barrera antes de convertirse en graduado. En este trabajo, se espera que el alumno demuestre las capacidades y conocimientos adquiridos a lo largo de su formación universitaria desde una perspectiva práctica y más cercana a lo que se espera que realice una vez comience su andadura por el mundo laboral.

    \paragraph{}
    Estas ideas son de carácter general y no dependen de la titulación que se esté realizando. Sin embargo, el trabajo de fin de grado depende fuertemente de la titulación a la cual se refiera. Es trivial entender que un alumno que haya cursado estudios de \emph{Filología Hispánica} no tenga nada que ver con el de un alumno que cuyos estudios estén referidos a otros ámbitos del conocimiento como la \emph{Física}, puesto que sus competencias son muy difierentes. Todos ellos tendrán una base común, realizando una introducción previa al tema que pretenden desarrollar, posiblemente describiendo el contexto histórico, seguidamente desarrollando el tema principal para finalmente llegar a unas conclusiones específicas.

    \paragraph{}
    En este caso, este \emph{Trabajo de Fin de Grado} se refiere a la titulación del \emph{Grado} en \emph{Ingeniería Informática} impartido en la \emph{Escuela Técnica Superior de Ingeniería Informática} de \emph{Valladolid}, dependiente de la \emph{Universidad de Valladolid}. Por esta razón, el trabajo será referido completamente al ámbito de la informática. Sin embargo, en este caso sucede una característica similar al descrito en el párrafo anterior. En esta titulación existen 3 menciones (o especialidades) que tratan de segregar las competencias que se enseñan, de tal manera que los alumnos puedan llegar a un mayor grado de conocimiento en la disciplina que más prefieran.

    \paragraph{}
    La razón de dicha separación durante el segundo ciclo de la titulación de grado se debe al amplísimo crecimiento que se está llevando a cabo en los últimos años, de tal manera que a pesar de haber una serie de conocimientos comunes que todo \emph{Ingeniero Informático} debe conocer, llega un punto en que la diversificación de áreas dificultan la tarea de adquicisión de todos aquellos conceptos en profundidad. Por tanto, parece apropiado dividir dichas disciplinas en ramas separadas. En el caso de la titulación para la cual se realiza este trabajo, existen 3 menciones: \emph{Tecnologías de Información}, \emph{Ingeniería de Software} y \emph{Computación}.

    \paragraph{}
    En este documento no se describirán cada una de ellas, ni se realizará una diferenciación de las mismas, puesto que esto ya puede ser consultado a través de la página web de la \emph{Escuela Técnica Superior de Ingeniería Informática} de \emph{Valladolid} a través de \url{https://www.inf.uva.es/}. Sin embargo, si que se indicará que este trabajo ha sido realizado tras haber seguido la mención en \emph{Computación}, la cual se refiere a los aspectos más teorícos, matemáticos y abstractos de la informática, tratando de dejar de lado el contexto del problema para centrarse en la búsqueda eficiente de la solución.

    \paragraph{}
    La razón por la cual se indica dicha explicación acerca de las distintas menciones sobre las que completar la titulación de \emph{Grado} en \emph{Ingeniería Informática}, así como el ejemplo inicial acerca de la diferenciación a nivel de contenido entre distintos trabajos de \emph{Fin de Grado} dependiendo de la titulación se debe a lo siguiente:

    \paragraph{}
    Este trabajo se ha pretendido focalizar en el estudio de \emph{Algoritmos para Big Data, Grafos y PageRank} desde una perspectiva mayoritariamente teórica, dejando de lado aspectos y cuestiones referidas a la implementación de los mismos. A pesar de que se ha realizado una implementación en el trabajo, esta ha sido de carácter ilustrativo, teniendo que requirir de trabajo adicional si pretender convertirse en una implementación adecuada para ser usada en entornos de producción.

    \paragraph{}
    Esto se contrapone con los temás que se desarrollan comúnmente para los estudios en cuestión, que generalmente basan un mayor esfuerzo en la parte de la  implementación para llegar en muchas ocasiones a un producto o servicio final. Esto se debe a las competencias desarrolladas, que se centran en ese tipo de actividades. Sin embargo, esto se contrapone con las competencias adquiridas durante el desarrollo de la mención en \emph{Computación}, la cual, tal y como se ha indicado anteriormente, se centra mayoritariamente en el apartado teórico y matemático de la resolución de problemas de manera eficiente.

    \paragraph{}
    Una vez realizada dicha distinción, ya se está en condiciones de comenzar a tratar el tema sobre el cual trata este \emph{Trabajo de Fin de Grado}. Para ello, se ha creido conveniente realizar una descripción acerca de las ideas iniales que se tenían antes de comenzar el trabajo, las cuales son drásticamente diferentes de las que se tiene una vez se ha finalizado el mismo. Esto se realiza en la sección \ref{sec:introduction_initial_ideas}. A continuación se hablará acerca de las motivaciones tanto personales como académicas que han propiciado la selección de dicho tema en la sección \ref{sec:introduction_motivation}. Posteriormente, en las secciones \ref{sec:introduction_big_data} y \ref{sec:introduction_graphs} se realiza una descripción superficial acerca del \emph{Big Data} y la modelización de \emph{Grafos} respectivamente, puesto que son los temas principales de dicho trabajo. Por último, en la sección \ref{sec:introduction_goals} se indican una serie de objetivos que se han pretendido conseguir mediante la realización de este trabajo.

    \section{Ideas Iniciales}
    \label{sec:introduction_initial_ideas}

      \paragraph{}
      [TODO]

    \section{Motivación}
    \label{sec:introduction_motivation}

      \paragraph{}
      [TODO]

    \section{Big Data}
    \label{sec:introduction_big_data}

      \paragraph{}
      El procesamiento de cantidades masivas de información presenta un gran reto a nivel computacional, debido a un elevado coste originado por el gran tamaño en la entrada. Para solventar dicha problemática, se prefieren algoritmos que posean un orden de complejidad sublineal ($o(N)$) sobre todo en espacio. Dichas técnicas se llevan a cabo sobre paradigmas de computación paralela, lo que permite aprovechar en mayor medida las restriciciones a nivel de hardware.

      \subsection{Algoritmos para Streaming}

        \paragraph{}
        Los \emph{Algoritmos para Streaming} se caracterizan por procesar las instancias del conjunto de datos secuencialmente e imponen como restricción que el orden de dicha operación sea irrelevante para el resultado final. La ventaja que presentan respecto de otras alternativas en tiempo real, como los \emph{Algoritmos Online}, es la utilización de propiedades estadísticas (se enmarcan por tanto, dentro de los \emph{Algoritmos Probabilísticos}) para reducir su coste, lo que por contra, añade una determinada tasa de error. El descubrimiento de métodos altamente eficientes para estimar los \emph{Momentos de Frecuencia} ha marcado un gran hito dentro de esta categoría algorítmica.

      \subsection{Estructuras de Datos de Resumen}

        \paragraph{}
        Para reducir el coste derivado de la obtención de resultados valiosos sobre conjuntos masivos de datos, es necesario apoyarse en diferentes estructuras que los sinteticen, de manera que el coste de procesamiento a partir de estas estructuras se convierta en una tarea mucho más asequible. Se utilizan sobre conjuntos de datos de distínta índole, como \emph{streamings en tiempo real}, \emph{bases de datos estáticas} o \emph{grafos}. Existen distintas técnicas como \emph{Sampling}, \emph{Histogram}, \emph{Wavelets} o \emph{Sketch}. A continuación se realiza una breve descripción acerca de esta última técnica.

        \subsection{Sketch}

          \paragraph{}
          Son estructuras de datos que se basan en la idea de realizar sobre cada una de las instancias del conjunto de datos la misma operación (lo que permite su uso en entornos tanto estáticos como dinámicos) para recolectar distintas caracteríscas. Destacan los \emph{Sketches lineales}, que permiten su procesamiento de manera distribuida. Para mantener estas estructuras se utilizan \emph{Algoritmos para Streaming}, puesto que se encajan perfectamente en el contexto descrito. Los \emph{Sketches} permiten realizar distintas preguntas sobre propiedades estadísticas referentes al conjunto de datos. Los ejemplos más destacados son: \emph{Count-Sketch}, \emph{CountMin-Sketch}, \emph{AMS Sketch}, \emph{HyperLogLog}, etc.

      \subsection{Redución de la Dimensionalidad}

        \paragraph{}
        Los algoritmos que utilizan técnicas de reducción de dimensionalidad se basan en la intuicción originada a partir del lema de \emph{Johnson–Lindenstrauss}, que demuestra la existencia de funciones para la redución de la dimensión espacial con un ratio de distorsión acotado. Estas técnicas son utilizadas en algoritmos para la \emph{busqueda de los vecinos más cercanos}, la \emph{multiplicación aproximada de matrices} o el aprendizaje mediante \emph{Manifold Leaning}.

      \subsection{Paralelización a gran Escala}

        \paragraph{}
        El paradigma de alto nivel sobre el que se lleva a cabo el procesamiento de conjuntos de datos de gran escala se apoya fuertemente en técnicas de paralelización. La razón se debe al elevado tamaño de la entrada, que no permite su almacenamiento en la memoria de un único sistema.

        \subsection{Modelo MapReduce}

          \paragraph{}
          El modelo \emph{MapReduce} ha sufrido un crecimiento exponencial en los últimos años debido a su alto grado de abstracción, que oculta casi por completo cuestiones relacionadas con la implementación de bajo nivel al desarrollador, y su capacidad para ajustarse a un gran número de problemas de manera eficiente.

      \subsection{Técnicas de Minería de Datos}

        \paragraph{}
        Una de las razones por las cuales es necesaria la investigación de nuevos algoritmos de carácter sublineal es la necesidad de obtención de información valiosa a partir de conjuntos masivos de datos. A este fenómeno se le denomina \emph{Minería de Datos}. Existen dos grandes categorías denominadas: \emph{Clasificación} (determinar una clase de pertenencia) y \emph{Regresión} (determinar un valor continuo). Para ello, se utilizan distintas técnicas como: \emph{Árboles de Decisión}, \emph{Métodos Bayesianos}, \emph{Redes Neuronales}, \emph{Máquinas de Vector Soporte}, \emph{Manifold Leaning}, etc.

    \section{Grafos}
    \label{sec:introduction_graphs}

      \paragraph{}
      [TODO]

    \section{Objetivos}
    \label{sec:introduction_goals}

      \paragraph{}
      [TODO]

\end{document}
