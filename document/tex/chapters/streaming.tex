% !TEX root = ../../document.tex

\documentclass{subfiles}

\begin{document}

  \chapter{Algoritmos para Streaming}
  \label{chapter:streaming}


    \section{Introducción}
    \label{sec:streaming_intro}

      \paragraph{}
      En este capítulo se trata de realizar una descripción en profundidad acerca de los \emph{Algoritmos en Streaming} desde una perspectiva tanto a teórica como práctica. Para ello se describirá el modelo de cómputo en que se enmarcan dichos algoritmos (Modelo en Streaming) en la sección \ref{sec:streaming_model} además de su estructura básica en la sección \ref{sec:streaming_structure}. El motivo de dicha descripción se debe a que los \emph{Algoritmos para Streaming} presentan un conjunto de peculiaridades respecto de la gran mayoría de algoritmos utilizados comunmente. [TODO introducir el resto de secciones]

      \paragraph{}
      Para realizar una primera aproximación acerca de en qué consiste esta categoría algorítmica es necesario realizar una diferenciación entre distintos conceptos relacionados con ella, que pueden producir confusiones debido a su similitud o abusos previos del lenguaje. Por lo tanto, a continuación se describen conceptos relacionados con los \emph{Algoritmos en Streaming} que permitirán introducir al lector en el contexto del problema. Además, se realiza una diferenciación acerca de los factores que se pretenden optimizar a partir de esta estrategia de diseño de algoritmos.


      \subsection{Computación en Tiempo Real}
      \label{sec:realtime_computing}
        \paragraph{}
        El primer concepto que se describe es \textbf{Computación en Tiempo Real}, que tal y cómo describen Shin y Ramanathan \cite{259423} se carácteriza por tres términos que se describen a continuación:

        \begin{itemize}

          \item \textbf{Tiempo}\emph{(time)}: En la disciplina de \emph{Computación en Tiempo Real} el tiempo de ejecucción de una determinada tarea es especialmente crucial para garantizar el correcto desarrollo del cómputo, debido a que se asume un plazo de ejecucción permitido, a partir del cual la solución del problema deja de tener un valor óptimo. Shin y Ramanathan\cite{259423} diferencian entre tres categorías dentro de dicha restricción, a las cuales denominan \emph{hard}, \emph{firm} y \emph{soft}, dependiendo del grado de relajación de la misma.

          \item \textbf{Confiabilidad}\emph{(correctness)}: Otro de los puntos cruciales en un sistema de \emph{Cómputación en Tiempo Real} es la determinación de una unidad de medida o indicador acerca de las garantias de una determinada solución algorítmica para cumplir lo que promete de manera correcta en el tiempo esperado.

          \item \textbf{Entorno}\emph{(environment)}: El último factor que indican Shin y Ramanathan\cite{259423} para describir un sistema de \emph{Computación en Tiempo Real} es el entorno del mismo, debido a que este condiciona el conjunto de tareas y la periodicidad en que se deben llevar a cabo. Debido a esta razón, realizan una diferenciación entre
          \begin{enumerate*}[label=\itshape\alph*\upshape)]
  					\item tareas periódicas \emph{periodic tasks} las cuales se realizan secuencialmente a partir de la finalización de una ventana de tiempo, y
  					\item tareas no periódicas \emph{periodic tasks} que se llevan a cabo debido al suceso de un determinado evento externo.
  				\end{enumerate*}

        \end{itemize}


      \subsection{Problemas Dinámicos}
      \label{sec:dynamic_problems}

        \paragraph{}
        Una vez completada la descripción acerca de lo que se puede definir como \emph{Computación en Tiempo Real}, conviene realizar una descripción desde el punto de vista de la \emph{teoría de complejidad computacional}. Para definir este tipo de problemas, se utiliza el término \emph{problemas dinámicos}, los cuales consisten en aquellos en los cuales es necesario recalcular su solución conforme el tiempo avanza debido a variaciones en los parámetros de entrada del problema (Nótese que dicho término no debe confundirse con la estrategia de \emph{programación dinámica} para el diseño de algoritmos). Existen distintas vertientes dependiendo del punto de vista desde el que se estudien, tanto de la naturaleza del problema (soluciones dependientes temporalmente unas de otras o soluciones aisladas) como de los parámetros de entrada (entrada completa en cada nueva ejecución o variación respecto de la anterior). Los \emph{Algoritmos para Streaming} están diseñados para resolver \emph{problemas dinámicos}, por lo que en la sección \ref{sec:streaming_model}, se describe en profundidad el modelo en que se enmarcan.

        \paragraph{}
        A continuación se indican los principales indicadores utilizados para describir la complejidad de una determinada solución algorítmica destinada a resolver un problema de dicha naturaleza:

        \begin{itemize}
          \item Espacio: Cantidad de espacio utilizado en memoria durante la ejecución del algoritmo.
          \item Inicialización: Tiempo necesario para la inicialización del algoritmo.
          \item Procesado: Tiempo necesario para procesar una determinada entrada.
          \item Pregunta[TODO Buscar mejor palabra]: Tiempo necesario para procesar la solución a partir de los datos de entrada procesados hasta el momento.
        \end{itemize}


      \subsection{Algoritmos Online vs Algoritmos Offline}

        \paragraph{}
        Una vez descrita la problemática de \emph{Computación en Tiempo Real} en la sección \ref{sec:realtime_computing} y la categoría de \emph{Problemas Dinámicos} en la sección \ref{sec:dynamic_problems}, en esta sección se pretende ilustrar la diferencia entre los \emph{Algoritmos Online} y los \emph{Algoritmos Offline}. A continuación se muestran las carácterísticas de cada subgrupo:

        \begin{itemize}
          \item \textbf{Algoritmos Online}: [TODO]

          \item \textbf{Algoritmos Offline}: [TODO]
        \end{itemize}

        \paragraph{}
        [TODO hablar acerca del "análisis competitivo"]


    \section{Modelo en Streaming}
    \label{sec:streaming_model}

      \paragraph{}
      [TODO]

      \subsection{Modelo de Serie Temporal}
        \paragraph{}
        [TODO]

      \subsection{Modelo de Caja Registradora}
        \paragraph{}
        [TODO]

      \subsection{Modelo de Molinete}
        \paragraph{}
        [TODO]

    \section{Estructura básica}
    \label{sec:streaming_structure}

      \paragraph{}
      [TODO]

\end{document}
