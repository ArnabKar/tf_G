% !TEX root = ../../document.tex

\documentclass{subfiles}

\begin{document}

  \chapter{Algoritmos para Streaming}
  \label{chapter:streaming}


    \section{Introducción}
    \label{sec:streaming_intro}

      \paragraph{}
      En este capítulo se trata de realizar una descripción en profundidad acerca de los \emph{Algoritmos en Streaming} desde una perspectiva tanto a teórica como práctica. Para ello se describirá el modelo de cómputo en que se enmarcan dichos algoritmos (Modelo en Streaming) en la sección \ref{sec:streaming_model} además de su estructura básica en la sección \ref{sec:streaming_structure}. El motivo de dicha descripción se debe a que los \emph{Algoritmos para Streaming} presentan un conjunto de peculiaridades respecto de la gran mayoría de algoritmos utilizados comunmente. [TODO introducir el resto de secciones]

      \paragraph{}
      Para realizar una primera aproximación acerca de en qué consiste esta categoría algorítmica es necesario realizar una diferenciación entre distintos conceptos relacionados con ella, que pueden producir confusiones debido a su similitud o abusos previos del lenguaje. Por lo tanto, a continuación se describen conceptos relacionados con los \emph{Algoritmos en Streaming} que permitirán introducir al lector en el contexto del problema. Además, se realiza una diferenciación acerca de los factores que se pretenden optimizar a partir de esta estrategia de diseño de algoritmos.


      \subsection{Computación en Tiempo Real}
      \label{sec:realtime_computing}
        \paragraph{}
        El primer concepto que se describe es \textbf{Computación en Tiempo Real}, que tal y cómo describen Shin y Ramanathan \cite{259423} se carácteriza por tres términos que se describen a continuación:

        \begin{itemize}

          \item \textbf{Tiempo}\emph{(time)}: En la disciplina de \emph{Computación en Tiempo Real} el tiempo de ejecucción de una determinada tarea es especialmente crucial para garantizar el correcto desarrollo del cómputo, debido a que se asume un plazo de ejecucción permitido, a partir del cual la solución del problema deja de tener un valor óptimo. Shin y Ramanathan\cite{259423} diferencian entre tres categorías dentro de dicha restricción, a las cuales denominan \emph{hard}, \emph{firm} y \emph{soft}, dependiendo del grado de relajación de la misma.

          \item \textbf{Confiabilidad}\emph{(correctness)}: Otro de los puntos cruciales en un sistema de \emph{Cómputación en Tiempo Real} es la determinación de una unidad de medida o indicador acerca de las garantias de una determinada solución algorítmica para cumplir lo que promete de manera correcta en el tiempo esperado.

          \item \textbf{Entorno}\emph{(environment)}: El último factor que indican Shin y Ramanathan\cite{259423} para describir un sistema de \emph{Computación en Tiempo Real} es el entorno del mismo, debido a que este condiciona el conjunto de tareas y la periodicidad en que se deben llevar a cabo. Debido a esta razón, realizan una diferenciación entre:
          \begin{enumerate*}[label=\itshape\alph*\upshape)]
  					\item tareas periódicas \emph{periodic tasks} las cuales se realizan secuencialmente a partir de la finalización de una ventana de tiempo, y
  					\item tareas no periódicas \emph{periodic tasks} que se llevan a cabo debido al suceso de un determinado evento externo.
  				\end{enumerate*}

        \end{itemize}


      \subsection{Problemas Dinámicos}
      \label{sec:dynamic_problems}

        \paragraph{}
        Una vez completada la descripción acerca de lo que se puede definir como \emph{Computación en Tiempo Real}, conviene realizar una descripción desde el punto de vista de la \emph{teoría de complejidad computacional}. Para definir este tipo de problemas, se utiliza el término \emph{problemas dinámicos}, los cuales consisten en aquellos en los cuales es necesario recalcular su solución conforme el tiempo avanza debido a variaciones en los parámetros de entrada del problema (Nótese que dicho término no debe confundirse con la estrategia de \emph{programación dinámica} para el diseño de algoritmos). Existen distintas vertientes dependiendo del punto de vista desde el que se estudien, tanto de la naturaleza del problema (soluciones dependientes temporalmente unas de otras o soluciones aisladas) como de los parámetros de entrada (entrada completa en cada nueva ejecución o variación respecto de la anterior). Los \emph{Algoritmos para Streaming} están diseñados para resolver \emph{problemas dinámicos}, por lo que en la sección \ref{sec:streaming_model}, se describe en profundidad el modelo en que se enmarcan.

        \paragraph{}
        A continuación se indican los principales indicadores utilizados para describir la complejidad de una determinada solución algorítmica destinada a resolver un problema de dicha naturaleza:

        \begin{itemize}
          \item Espacio: Cantidad de espacio utilizado en memoria durante la ejecución del algoritmo.
          \item Inicialización: Tiempo necesario para la inicialización del algoritmo.
          \item Procesado: Tiempo necesario para procesar una determinada entrada.
          \item Pregunta[TODO Buscar mejor palabra]: Tiempo necesario para procesar la solución a partir de los datos de entrada procesados hasta el momento.
        \end{itemize}


      \subsection{Algoritmos Online vs Algoritmos Offline}

        \paragraph{}
        Una vez descrita la problemática de \emph{Computación en Tiempo Real} en la sección \ref{sec:realtime_computing} y la categoría de \emph{Problemas Dinámicos} en la sección \ref{sec:dynamic_problems}, en esta sección se pretende ilustrar la diferencia entre los \emph{Algoritmos Online} y los \emph{Algoritmos Offline}. Para ello, se ha seguido la diferenciación propuesta por Karp \cite{Karp:1992:OAV:645569.659725}, en la cual se plantea el problema de la siguiente manera (Se utilizará la misma notación que sigue Muthukrishnan\cite{Muthukrishnan:2005:DSA:1166409.1166410} para tratar mantener la consistencia durante todo el documento): Sea $A$ el conjunto de datos o eventos de entrada, siendo cada $A[i]$ el elemento \emph{i-ésimo} del conjunto. En el caso de los \emph{Algoritmos Online} supondremos que es el elemento recibido en el instante \emph{i}. A continuación se muestran las características de cada subgrupo:

        \begin{itemize}
          \item \textbf{Algoritmos Offline}: Esta categoría contiene todos los algoritmos que realizan el cómputo suponiendo el acceso a cualquier elemento del conjunto de datos $A$ durante cualquier momento de su ejecución. Además, en esta categoría se impore la restricción de que el $A$ debe ser invariante respecto del tiempo, lo que impone que para la adaptación del resultado a cambios, este tenga que realizar una nueva ejecución desde su estado inicial. Nótese que por tanto, dentro de este grupo se engloba la mayoría de algoritmos utilizados comunmente.

          \item \textbf{Algoritmos Online}: Son aquellos que calculan el resultado a partir de una secuencia de sucesos $A[i]$, los cuales generan un resultado dependiente del valor, y posiblemente de los sucedidos anteriormente. A partir de dicha estrategia, se añade una componente dinámica, la cual permite que tamaño del conjunto de datos de entrada $A$ no tenga impuesta una restricción acerca de su longitud. Por contra, en este modelo no se permite conocer el cuceso $A[i+1]$ en el momento $i$.  Esto encaja perfectamente en el modelo que se describirá en la sección \ref{sec:streaming_model}.

        \end{itemize}

        \paragraph{}
        Según la diferenciación que se acaba de describir, estas dos estrategias de diseño de algoritmos encajan en disciplinas distintas de diseño de algoritmos, teniendo una gran ventaja a nivel de eficiencia en el caso estático los \emph{Algoritmos Offline}, pero quedando prácticamente inutilizables cuando la computación es en tiempo real, donde es mucho más apropiado el uso de estrategias de diseño de \emph{Algoritmos Online}.

        \paragraph{}
        Como medida de eficiencia para los \emph{Algoritmos Online}, Karp \cite{Karp:1992:OAV:645569.659725} propone el \textbf{Ratio Competitivo}, el cual se define como la cota inferior del coste de cualquier nueva entrada con respecto de la que tiene menor coste. Sin embargo, dicha medida de eficiencia no es comúnmente utilizada en el caso de los \emph{Algoritmos para Streaming} por la componente estocástica de los mismos, para los cuales son más apropiadas medidas probabilistas. A continuación se describen las ventajas de estos respecto de su vertiente estática.


      \subsection{Algoritmos Probabilistas}

        \paragraph{}
        Los \emph{Algoritmos Probabilistas} son una estrategia de diseño que emplea en un cierto grado de aleatoriedad en alguna parte de su lógica. Estos utilizan distribuciones uniformes de probabilidad para tratar de conseguir un incremento del rendimiento en su caso promedio. A continuación se describen los dos tipos de algoritmos probabilísticos según la clasificación realizada por Babai \cite{Babai79monte-carloalgorithms}:

        \begin{itemize}

          \item \textbf{Algoritmos Las Vegas}: Devuelven un resultado incorrecto con una determinada probabilidad, pero avisan del resultado incorrecto cuando esto sucede. Para contrarrestrar este suceso basta llevar a cabo una nueva ejecución del algoritmo, lo cual tras un número indeterminado de ejecuciones produce un resultado válido.

          \item \textbf{Algoritmos Monte Carlo}: Fallan con un cierto grado de probabilidad, pero en este caso no avisan del resultado incorrecto. Por lo tando, lo único que se puede obtener al respecto es una indicador de la estimación del resultado correcto hacia la que converge tras varias ejecuciones. Además, se asegura una determinada cota del error $\epsilon$, que se cumple con probabilidad $\delta$.

        \end{itemize}

        \paragraph{}
        La razón anecdótica por la cual Babai \cite{Babai79monte-carloalgorithms} decidió denominar dichas categorías de algoritmos de esta manera se debe a lo siguiente (teniendo en cuenta el contexto de lengua inglesa): cuando se va a un casino en \emph{Las Vegas} y se realiza una apuesta el \emph{croupier} puede decir si se ha ganado o perdido porque habla el mismo idioma. Sin embargo, si sucede la misma situación en \emph{Monte Carlo}, tan solo se puede conocer una medida de probabilidad debido a que en este caso el \emph{croupier} no puede comunicarlo por la diferencia dialéctica.


      \subsection{Algoritmos Online Probabilistas vs Deterministas}

        \paragraph{}
        La idea subyacente acerca del diseño de los \emph{Algoritmos Online} es la mejora de eficiencia con respecto a sus homónimos estáticos cuando el conjunto de valores de entrada es dependiente de los resultados anteriores. Sin embargo, existen casos en que la frecuencia de ejecución del algoritmo, debido a una alta tasa de llegada de sucesos de entrada, las soluciones deterministas se convierten en alternativas poco escalables.

        \paragraph{}
        Dicha problemática se ha incrementado de manera exponencial debido al avance tecnológico y la gran cantidad de información que se está generando en la actualidad. Este fenómeno ha convertido en algo necesario el diseño de estrategias basadas en sucesos probabilísticos que reduzcen en gran medida el coste computacional eliminando el determinismo de la solución.


    \section{Modelo en Streaming}
    \label{sec:streaming_model}

      \paragraph{}
      En esta sección se describen los aspectos formales del \emph{Modelo en Streaming}. Para ello se ha seguido la representación definida por Muthukrishnan \cite{Muthukrishnan:2005:DSA:1166409.1166410}. Lo primero por tanto, es definir lo que es un flujo de datos o \emph{Data Stream} como una \say{secuencia de señales digitalmente codificadas utilizadas para representar una transmisión de información} \cite{ITS-def-data-stream}. Muthukrishnan \cite{Muthukrishnan:2005:DSA:1166409.1166410} hace una aclaración sobre dicha definición y añade la objeción de que los datos de entrada deben tener un ritmo elevado de llegada. Debido a esta razón existe complejidad a tres niveles:

      \begin{itemize}

        \item \textbf{Transmisión}: Ya que debido a la alta tasa de llegada es necesario diseñar un sistema de interconexiones que permita que no se produzcan congestiones debido a la obtención de los datos de entrada.

        \item \textbf{Computación}: Puesto que la tarea de procesar la gran cantidad de información que llega por unidad de tiempo produce cuellos de botella en el planteamiento. Por lo que es necesario implementar técnicas algorítmicas con un reducido nivel de complejidad computacional que para contrarrestar dicha problemática.

        \item \textbf{Almacenamiento}: Debido a la gran cantidad de datos que se presentan en la entrada, deben existir técnicas que permitan almacenar dicha información de manera eficiente. Esto puede ser visto desde dos puntos de vista diferences: \begin{enumerate*}[label=\itshape\alph*\upshape)]
          \item tanto desde el punto de vista del espacio, tratando de minimizar el tamaño de los datos almacenados, maximizando la cantidad de información que se puede recuperar de ellos,
          \item como desde el punto de vista del tiempo necesario para realizar operaciones de búsqueda, addición, eliminación o edición.
        \end{enumerate*}

      \end{itemize}

      \paragraph{}
      [TODO ejemplo de los tres niveles de complejidad a partir del análisis meteorológico ]

      \paragraph{}
      Una vez descritos los niveles de complejidad a los que es necesario enfrentarse, en los apartados \ref{sec:streaming_time_series}, \ref{sec:streaming_cash_register} y \ref{sec:streaming_turnstile} se realiza una descripción acerca de los distintos modelos que propone Muthukrishnan \cite{Muthukrishnan:2005:DSA:1166409.1166410}. La notación descrita en dichos apartados será seguida durante el resto del capítulo. Para ello nos basaremos en el siguiente formalismo:

      \paragraph{}
      Sea $a_1, a_2,...$ el conjunto de posibles elementos de entrada con cardinalidad $N$ (posiblemente no acotada tal y como se verá en el caso de las \emph{Series Temporales} (\ref{sec:streaming_time_series})), se debe cumplir que $\forall i, \forall j / i \neq j \implies a_i \neq a_j$. Se ha decidido introducir el símbolo \say{$\cdot$} como comodín siguiendo la misma interpretación que utilizan los sistemas \emph{UNIX} en el contexto de las \emph{Expresiones regulares}\cite{gnu:regular-expresions-fundamental-structure}, de tal manera que $a_{\cdot}$ indica que el elemento al que nos estamos refiriendo puede ser cualquier valor.

      \paragraph{}
      Lo siguiente es añadir la componente dinámica al formalismo. Por tanto, diremos que $a_\cdot[t]$ se refiere al elemento recibido en el instante de tiempo $t$. La restricción más relevante que se impone a un \emph{Modelo en Streaming} es la siguiente: Sea $a_\cdot[1],a_\cdot[2],...,a_\cdot[t],... $ un flujo de entrada (\emph{Input Stream}), de tal manera que cada elemento debe tener un orden de llegada secuencial respecto de $t$. Esto quiere decir que el elemento previo a la llegada de $a_i$ debe ser $a_{i-1}$ y por inducción, el siguiente será $a_{i+1}$.


      \subsection{Modelo de Serie Temporal}
      \label{sec:streaming_time_series}

        \paragraph{}
        El \emph{Modelo de Serie Temporal} o \emph{Time Series Model}

      \subsection{Modelo de Caja Registradora}
      \label{sec:streaming_cash_register}

        \paragraph{}
        [TODO]

      \subsection{Modelo de Molinete}
      \label{sec:streaming_turnstile}

        \paragraph{}
        [TODO]

    \section{Estructura básica}
    \label{sec:streaming_structure}

      \paragraph{}
      [TODO]

    \section{Medidas de Análisis}
    \label{sec:streaming_analysis}

      \paragraph{}
      [TODO]


\end{document}
