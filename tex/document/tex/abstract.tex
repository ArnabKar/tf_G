% !TEX root = ../document.tex

\documentclass{subfiles}


\newenvironment{abstractpage}
  {\cleardoublepage\vspace*{\fill}\thispagestyle{empty}}
  {\vfill\cleardoublepage}
\newenvironment{abstract-lang}[1]
  {\bigskip\selectlanguage{#1}%
   \begin{center}\bfseries\abstractname\end{center}}
  {\par\bigskip}


\begin{document}

  \begin{abstractpage}
    \addcontentsline{toc}{chapter}{\protect\numberline{}Resumen}
    \begin{abstract-lang}{english}
      This work consists of a study of a set of techniques and strategies related with algorithm's design, whose purpose is the resolution of problems on massive data sets, in an efficient way. This field is known as \emph{Algorithms for Big Data}. In particular, this work has studied the \emph{Streaming Algorithms}, which represents the basis of the data structures of sublinear order $o(n)$ in space, known as \emph{Sketches}. In addition, it has deepened in the study of problems applied to \emph{Graphs} on the \emph{Semi-Streaming} model. Next, the \emph{PageRank} algorithm was analyzed as a concrete case study. Finally, the development of a library for the resolution of graph problems, implemented on the top of the intensive mathematical computation platform known as \emph{TensorFlow} has been started.
    \end{abstract-lang}
    \begin{abstract-lang}{spanish}
      En este trabajo se ha realizado un estudio acerca de las diferentes técnicas y estrategias de diseño de algoritmos, pensadas para la resolución de problemas sobre conjuntos de datos de tamaño masivo, de manera eficiente. Este campo es conocido conocido como \emph{Algoritmos para Big Data}. En concreto, en este trabajo se ha profundizado en el estudio de los \emph{Algoritmos para Streaming}, que representan la base de las estructuras de datos de orden sublineal $o(n)$ en espacio, conocidas como \emph{Sketches}. Además, se ha profundizado en el estudio de problemas aplicados a \emph{Grafos} sobre el modelo en \emph{Semi-Streaming}. Seguidamente, se ha analizado el algoritmo \emph{PageRank} como caso concreto de estudio. Por último, se ha comenzado el desarrollo de una biblioteca para la resolución de problemas de grafos, implementada sobre la plataforma de cómputo matemático intensivo \emph{TensorFlow}.
    \end{abstract-lang}

    \centering
    Este trabajo ha sido publicado en: \url{https://github.com/garciparedes/tf_G}

  \end{abstractpage}

\end{document}
